\begin{frame}
    \frametitle{Deklarativní programování}

    \begin{itemize}
        \item Definice se různí
        \begin{itemize}
            \item např.: Deklarativní - když není imperativní
        \end{itemize}
        \item Imperativní programování je založeno na \emph{Turingových strojích}
        \begin{itemize}
            \item Matematický koncept
            \item Jednojádrový procesor, jehož registry nabývají konečný počet stavů = celkový stav
            \item Operuje na konečném počtu slov = hodnot v buňkách paměti = instrukcích
        \end{itemize}
        \item Deklarativní programování je založeno jiném modelu výpočtů
    \end{itemize}
\end{frame}

\begin{frame}
    \frametitle{Funkcionální programování}

    \begin{itemize}
        \item Založeno na \emph{Lambda kalkulu}
        \begin{itemize}
            \item Lambda kalkulus např. nemá datové typy či rekurzi
            \item Ekvivalentní s Turingovými stroji
        \end{itemize}
        \item Pro připomenutí: \emph{Haskell, Elm}
    \end{itemize}
\end{frame}

\begin{frame}
    \frametitle{Logické programování}

    \begin{itemize}
        \item Založeno na formální logice
        \item Logický program je sada logických vět v jazyce dané domény a následné dotazy na tato pravidla
        \item Programovací jazyky:
        \begin{itemize}
            \item \emph{Prolog}
            \item \emph{ASP} - answer set programming
            \item \emph{Datalog}
        \end{itemize}
    \end{itemize}
\end{frame}
