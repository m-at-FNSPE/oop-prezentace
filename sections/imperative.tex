\begin{frame}
    \frametitle{Imperativní programování}
    Program je napsaný jako posloupnost příkazů a určuje přesný způsob jak úlohu řešit. % Velmi blízké člověku, například kuchařky atd.

    Imperativní programování se dá rozdělit hlavně na tyto způsoby: 
    \begin{itemize}
        \item Objektové programování
        \item Procedurální programování
    \end{itemize}

    % Každý program je přeložen do strojového kódu, který je imperativní 
    
\end{frame}

\begin{frame}[fragile]
    \frametitle{Imperativní programování - Historie}
    První programovací jazyky byly strojové jazyky, ty se skládají z velmi jednoduchých instrukcí a dají se tedy považovat za imperativní jazyk.

\end{frame}

\begin{frame}[fragile]
% Z https://stackoverflow.com/questions/51433054/add-two-numbers-in-assembly

    \frametitle{Imperativní programování - Ukázka Assembleru}

    \newfontfamily\AssemblyFamily[]{DejaVu Sans Mono}
    \begin{lstlisting} [basicstyle=\linespread{0.9}\AssemblyFamily , language={[x86masm]Assembler}, numbers=left]
SECTION .data         ;
    extern printf     ;Procedura, vypisuje zásobník
    global main       ;
fmt:                  ;
    db "%d", 10, 0    ;
SECTION .text         ;
main:
    mov     eax, 14   ;Uloží 14 do registru eax
    mov     ebx, 10   ;Uloží 10 do registru eax
    add     eax, ebx  ;Sečte eax a ebx a uloží do eax
    push    eax       ;Pošle hodnotu eax na zásobník
    push    fmt       ;Pošle hodnotu fmt na zásobník
    call    printf    ;Zavolá proceduru printf
    mov     eax, 1    ;Uloží 1 do registru eax
    int     0x80      ;Vyvolá přerušení a ukončí program
    \end{lstlisting}
\end{frame}

\begin{frame}
    \frametitle{Objektové programování}
    

\end{frame}

\begin{frame}
    \frametitle{Objektové programování}
    
\end{frame}