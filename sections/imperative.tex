\begin{frame}
    \frametitle{Imperativní programování}
    Program je napsaný jako posloupnost příkazů a určuje přesný způsob jak úlohu řešit. % Velmi blízké člověku, například kuchařky atd.

    Imperativní programování se dá rozdělit hlavně na tyto způsoby: 
    \begin{itemize}
        \item Objektové programování
        \item Procedurální programování
    \end{itemize}

    % Každý program je přeložen do strojového kódu, který je imperativní 
    
\end{frame}

\begin{frame}[fragile]
    \frametitle{Imperativní programování - Historie}
    První programovací jazyky byly strojové jazyky. 

    Jejich nástupcem jsou jazyky symbolických adres (assembler). % Jeho hlavní výhoda oproti zdrojovému kódu je že kód lze jednodušeji modifikovat
    
    Oba dva druhy jazyků se skládají z velmi jednoduchých instrukcí a většinou se
    \footnote{Existují vyjímky: v moderní době jsou to hlavně FPGA (programovatelné hradlové pole) nebo historicky například tzv. lisp-machines} považují za imperativní jazyk.
\end{frame}


\begin{frame}[fragile]
    \frametitle{Imperativní programování - Historie} % Možná to že se vyznačují cykly, přiřazením a příkazy pro větvení, ale nevím
\end{frame}

\begin{frame}[fragile]
% Z https://stackoverflow.com/questions/51433054/add-two-numbers-in-assembly
    \frametitle{Imperativní programování - Ukázka Assembleru}

    % Utf-8 enkódování nefunguje v lstlisting, proto jsou v literate nadefinované znaky
    \newfontfamily\AssemblyFamily[]{DejaVu Sans Mono}
    \begin{lstlisting} [basicstyle=\linespread{0.9}\AssemblyFamily,
                        language={[x86masm]Assembler},
                        xleftmargin=-0.75cm,
                        extendedchars=true,
                        literate={^}{{\v{z}}}1{$}{{\v{c}}}1{*}{{\v{s}}}1{?}{{\v{r}}}1,
                        ]
SECTION .data       ;
    extern printf   ;Na$te proceduru
    global main     ;
fmt:                ;
    db "%d", 10, 0  ;
SECTION .text       ;
main:               ;Ozna$uje za$átek programu
    mov     eax, 14 ;Ulo^í 14 do registru eax
    mov     ebx, 10 ;Ulo^í 10 do registru eax
    add     eax, ebx;Ulo^í do eax sou$et eax a ebx 
    push    eax     ;Po*le hodnotu eax na zásobník
    push    fmt     ;Po*le hodnotu fmt na zásobník
    call    printf  ;Zavolá proceduru printf
    mov     eax, 1  ;Ulo^í 1 do registru eax
    int     0x80    ;P?eru*ením ukon$í program
    \end{lstlisting}
\end{frame}


\begin{frame}
    \frametitle{Procedurální programování}
    Toto paradigma vzniklo spolu se vznikem vyšších programovacích jazyků.
    
    Základem paradigmatu je rozdělení programů do procedur/funkcí za cílem získání větší modularity.

    Procedurální programovaní hodně zavisí na použití bloků a scope. %možná zmínit něco o tom jak funguje. call stack atd

    Mezi hlavní představitele tohoto paradigmatu se řadí: \break C, Pascal, Fortran, COBOL 


\end{frame}

\begin{frame}
    \frametitle{Objektové programování}
    
\end{frame}