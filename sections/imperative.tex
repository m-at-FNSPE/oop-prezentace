\begin{frame}
    \frametitle{Imperativní programování}
    Program je napsaný jako posloupnost příkazů a určuje přesný způsob jak úlohu řešit. % Velmi blízké člověku, například kuchařky atd.

    Imperativní programování se dá rozdělit hlavně na tyto způsoby: 
    \begin{itemize}
        \item Objektové programování
        \item Procedurální programování
    \end{itemize}

    % Každý program je přeložen do strojového kódu, který je imperativní 
    
\end{frame}

\begin{frame}[fragile]
    \frametitle{Imperativní programování - Historie}
    První programovací jazyky byly strojové jazyky. 

    Jejich nástupcem jsou jazyky symbolických adres (assembler). % Jeho hlavní výhoda oproti zdrojovému kódu je že kód lze jednodušeji modifikovat
    
    Oba dva druhy jazyků se skládají z velmi jednoduchých instrukcí a většinou se
    \footnote{Existují vyjímky: v moderní době jsou to hlavně FPGA (programovatelné hradlové pole) nebo historicky například tzv. lisp-machines} považují za imperativní jazyk.
\end{frame}

\begin{frame}[fragile]
% Z https://stackoverflow.com/questions/51433054/add-two-numbers-in-assembly
    \frametitle{Imperativní programování - Ukázka Assembleru}

    % Utf-8 enkódování nefunguje v lstlisting, proto jsou v literate nadefinované znaky
    \newfontfamily\AssemblyFamily[]{DejaVu Sans Mono}
    \begin{lstlisting} [basicstyle=\linespread{0.9}\AssemblyFamily,
                        language={[x86masm]Assembler},
                        xleftmargin=-0.75cm,
                        extendedchars=true,
                        literate={^}{{\v{z}}}1{$}{{\v{c}}}1{*}{{\v{s}}}1{?}{{\v{r}}}1,
                        ]
SECTION .data       ;
    extern printf   ;Na$te proceduru
    global main     ;
fmt:                ;
    db "%d", 10, 0  ;
SECTION .text       ;
main:               ;Ozna$uje za$átek programu
    mov     eax, 14 ;Ulo^í 14 do registru eax
    mov     ebx, 10 ;Ulo^í 10 do registru eax
    add     eax, ebx;Ulo^í do eax sou$et eax a ebx 
    push    eax     ;Po*le hodnotu eax na zásobník
    push    fmt     ;Po*le hodnotu fmt na zásobník
    call    printf  ;Zavolá proceduru printf
    mov     eax, 1  ;Ulo^í 1 do registru eax
    int     0x80    ;P?eru*ením ukon$í program
    \end{lstlisting}
\end{frame}

\begin{frame}
    \frametitle{Strukturované programování}
    Strukturované programování zakazuje používání příkazu goto.
    Místo toho mají mít řídící struktury jeden vstupní a výstupní bod\footnote{Toto není vždy pravda, například příkaz break a obsluha vyjímek}.


\end{frame}


\begin{frame}
    \frametitle{Procedurální programování}
    Toto paradigma vzniklo spolu se vznikem vyšších programovacích jazyků.
    
    Základem paradigmatu je rozdělení programů do procedur/funkcí za cílem získání větší modularity.

    Procedurální programovaní hodně zavisí na použití bloků a scope. %možná zmínit něco o tom jak funguje. call stack atd

    Mezi hlavní představitele tohoto paradigmatu se řadí: \break C, Pascal, Fortran, COBOL 


\end{frame}


\begin{frame}
    \frametitle{Objektově orientované programování}
    Dá se považovat za nástupce procedurálního programování. 
    
    V objektovém programování pracujeme s objekty, což jsou entity které obsahují data (atributy) a kód (metody)

    Pro objekty platí následující: % Přeformuluj

    \begin{itemize}
        \item Abstrakce
        \item Zapouzdření
        \item Kompozice
        \item Dědičnost
        \item Polymorfismus
    \end{itemize}
    
\end{frame}


\begin{frame}
    \frametitle{Objektově orientované programování} % Možná ne úplně správně
    Programovací jazyky které umožňují objektové programování můžeme rozčlenit do 2 skupin:

    \begin{itemize}
        \item Čistě objektové jazyky\begin{itemize}
            \item V těchto jazycích jsou všechny datové typy objekty a všechny operace se provadí za pomoci metod.
            \item Například Smalltalk nebo Ruby
        \end{itemize}
        \item  Hybridní nebo Objektově orientované jazyky\begin{itemize}
            \item V těchto jazycích většinou například primitivní datové typy nejsou objekty
            \item Sem řadíme například C++, Rust, Java atd. 
        \end{itemize}
    \end{itemize}
\end{frame}




\begin{frame}[fragile]
    \frametitle{Objektově orientované programování - ukázka Ruby\footnote{https://www.includehelp.com/ruby/print-power-of-a-number.aspx}}

\begin{lstlisting}[]
def pow(a,b)
    power=1
    for i in 1..b
        power=power*a
    end
    return power
end
puts "Enter Base:"
base=gets.chomp.to_i
puts "Enter exponent:"
expo=gets.chomp.to_i
puts "The power is #{pow(base,expo)}"  
\end{lstlisting}
\end{frame}

\begin{frame}
    \frametitle{OOP modely}
        Existují 2 modely objektového programování:
        \begin{enumerate}
            \item Model využívající třídy
            \item Model využívající prototypy
        \end{enumerate} 
\end{frame}

\begin{frame}
    \frametitle{OOP - Model využívající třídy}
    V tomto modelu uživatel definuje třídy které poté instancuje.

    Třídu můžeme považovat za návod jak třídu instancovat.

    Tento model je používaný např v C++, Javě.
\end{frame}

\begin{frame}
    \frametitle{OOP - Model využívající prototypy}
    V tomto modelu existují pouze objekty. %\footnote{https://developer.mozilla.org/en-US/docs/Web/JavaScript/Inheritance_and_the_prototype_chain}

    Objektům přiřazujeme metody a atributy a můžeme od nich odvozovat další objekty. Rodičovský objekt od kterého odvozujeme nazýváme prototyp.

    Jazyky, které tento model implementují jsou téměř vždy interpretované a používají dynamický typový systém.

    Hlavním představitelem tohoto modelu je JavaScript, nebo popřípadě Lua.

    %Zjisti jestli je tento model silnější než třídy


\end{frame}

\begin{frame}
    \frametitle{OOP - Model využívající prototypy}

    Jak funguje dědění?\break
    Existují 2 způsoby jak jazyky mohou dědění implementovat

    \begin{enumerate}
        \item Delegace \begin{itemize}
            \item Odvozený objekt obsahuje tzv. delegační ukazatel (delegation pointer) na předka.
            \item Pokud zavoláme metodu, objekt se ji nejdříve pokusí najít ve svých metodách, pokud ji neobsahuje tak se metodu pokusí najít postupně ve svých předcích.  
        \end{itemize}
        \item Zřetězení (Concatenation) \begin{itemize}
            \item Zřídka používaný přístup. Používá ho například jazyk Kevo
            \item Objekty neobsahují delegační ukazatel, místo toho je prototyp zkopírován.
            \item Hlavním rozdílem je, že změny v předcích se automaticky nepropagují do potomků
        \end{itemize}
    \end{enumerate}

\end{frame}


\begin{frame}[fragile]
    \frametitle{Objektově orientované programování - ukázka JavaScript \footnote{https://javascript.info/prototype-inheritance}   }
\begin{lstlisting}[]

let animal = {
  eats: true,
  walk() {alert("Animal walk");}};
let rabbit = {
  jumps: true,
  __proto__: animal};
let longEar = {
  earLength: 10,
  __proto__: rabbit};
longEar.walk(); // prints "Animal walk"
alert(longEar.jumps); // prints "true"

\end{lstlisting}
    
\end{frame}