\begin{frame}
    \frametitle{Jazykově orientované programování (Language-oriented programming)}

    V tomto paradigmatu jsou jednotlivé jazyky položené na stejnou úroveň jako objekty nebo jiné komponenty.

    Toto paradigma říká, že namísto toho aby programátor problémy řešil v obecných programovacích jazycích, tak programátor vytvoří takzvané doménově specifické jazyky (Domain-specific language) a problémy řeší za pomoci nich.

\end{frame}


\begin{frame}
    \frametitle{Doménově specifické jazyky}
    DSL jsou jazyky které jsou určené k řešení jednoho konkrétního problému.

    Tyto jazyky většinou nejsou turingovsky kompletní. Ale vývojem se mohou turingovsky kompletní stát (Například Perl).

    


\end{frame}

\begin{frame}
    \frametitle{Doménově specifické jazyky}

    Jedním ze speciálních případů DSL jsou markup jazyky (HTML, LaTeX, Markdown)



\end{frame}

\begin{frame}[fragile]
    \frametitle{Doménově specifické jazyky - Shader languagess}
    Jazyky jako GLSL (OpenGL Shading Language) určené pro psaní shaderů pro grafické karty.
    
    Například GLSL používá upravenou syntaxi jazyka C. Příklad\footnote{https://thebookofshaders.com/}:

    \newfontfamily\AssemblyFamily[]{DejaVu Sans Mono}
    \begin{lstlisting} [basicstyle=\linespread{0.9}\AssemblyFamily,
                        language={[x86masm]Assembler}]
#ifdef GL_ES
precision mediump float;
#endif

uniform float u_time;

void main() {
    gl_FragColor = vec4(1.0,0.0,1.0,1.0);
}
                        
    \end{lstlisting}




\end{frame}

\begin{frame}[fragile]
    \frametitle{Doménově specifické jazyky - query languages}
    Jazyky určené pro práci s databázemi, nejznámějším příkladem je SQL.

Příklad:

    \begin{lstlisting}[language=SQL]
SELECT * FROM Customers
WHERE Country='Czech Republic'; 
    \end{lstlisting}
    
    
\end{frame}

\begin{frame}
    \frametitle{Doménově specifické jazyky - Regular Expressions}


\end{frame}